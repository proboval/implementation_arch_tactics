\chapter{Introduction}
\label{chap:intro}
\chaptermark{Optional running chapter heading}
\section{Background}
\label{sec:section}

Modern software systems increasingly rely on complex backend architectures that must remain maintainable throughout the development period. As systems grow in size and functionality, their internal structure tends to grow worse, leading to increased modification costs and technical debt \cite{molnar2020study}. Maintainability, defined in ISO/IEC 25010 as the ability of a system to be modified effectively \cite{ISO25010}, is therefore one of the most important quality attributes affecting the long-term sustainability of software.

One of the mechanisms for improving and maintaining maintainability at the architectural level is the use of architectural tactics -- reusable architectural building blocks providing general architectural solutions for common issues pertaining to quality attributes \cite{kim_quality-driven_2009}. Although these tactics are well described in the software architecture literature, their practical application in existing, imperfect, or evolving codebases remains inconsistent. Many real-world projects lack a clear architectural structure or have accumulated architectural erosion over years of development, making it unclear how the targeted implementation of tactics affects maintainability in practice \cite{kassab_software_2018, perry_foundations_1992}.

At the same time, recent advances in large language models (LLMs) have opened up new possibilities for automatic program transformation. LLMs demonstrate the ability to refactor code, restructure modules, introduce design patterns, and enforce architectural rules, often while preserving system behavior \cite{depalma_exploring_2024}. This ability raises an important question: can LLMs be used to implement architectural tactics in existing backend systems, and if so, how will this affect maintainability?

Understanding this relationship is particularly relevant for open source software, where architecture quality varies significantly and where maintainability issues often arise due to rapid, community-driven development \cite{kassab_software_2018}. By using LLM to implement architectural tactics in real GitHub projects, researchers can empirically evaluate how structural improvements affect maintainability metrics, architectural consistency, and technical debt reduction.

The goal of this thesis is to research whether automated integration of architectural tactics using LLM can lead to measurable improvements in the maintainability of existing backend architectures.

\section{Research objectives}
\label{sec:section}

To address the above issue, this thesis sets the following research objectives:
\begin{enumerate}
    \item To identify a set of architectural tactics that directly influence maintainability as defined by ISO/IEC 25010.

    \item Implement selected architectural tactics in existing open source backend projects using LLM to perform architectural transformations.

    \item Evaluate the impact of LLM-generated architectural changes on maintainability using a combination of code-level and architecture-level metrics, static analysis tools, and architecture erosion indicators.

    \item Compare maintainability before and after transformations using LLM, determining whether the tactics applied lead to measurable improvements.

    \item Evaluate both the advantages and limitations of using LLM for architecture-level modifications, including their consistency, correctness, and ability to preserve behavior.

    \item Formulate recommendations and guidelines for applying architectural tactics using LLM in real-world backend systems.
\end{enumerate}
